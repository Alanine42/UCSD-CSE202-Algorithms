\newpage
\problem{3: Scheduling} % {10+10+10+20=50}

\problemdes

Consider the following scheduling problem. You are given a set of $n$ jobs, each of which has a time requirement $t_i$. Each job can be done on one of two identical machines. The objective is to minimize the total time to complete all jobs, i.e., the maximum over the two machines of the total time of all jobs scheduled on the machine. A greedy heuristic would be to go through the jobs and schedule each on the machine with the least total work so far.

\subproblem{Subproblem 1}

Give an example (with the items sorted in decreasing order) where this heuristic is not optimal.

\subsolution{Solution 1}

\subsolution{High-level description}

% \subsolution{Pseudo Code}

\subsolution{Correctness}

\subsolution{Time complexity}


\subproblem{Subproblem 2}

Assume the jobs are sorted in decreasing order of time required. Show as tight a bound as possible on the approximation ratio for the greedy heuristic. A ratio of $7/6$ or better would get full credit. A ratio worse than $7/6$ might get partial credit.

\subsolution{Solution 2}

\subsolution{High-level description}

% \subsolution{Pseudo Code}

\subsolution{Correctness}

\subsolution{Time complexity}


