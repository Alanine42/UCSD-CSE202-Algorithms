\newpage
\problem{1: Hamiltonian path} % {10+10+10+20=50}

\problemdes

Suppose we are given a directed graph $G = (V,E)$, with $V = \{v_1,v_2, \dots,v_n\}$, and we want to decide whether $G$ has a Hamiltonian path from $v_1$ to $v_n$. (That is, is there a path in $G$ that goes from $v_1$ to $v_n$, passing through every other vertex exactly once?)

Since the Hamiltonian Path Problem is NP-complete, we do not expect that there is a polynomial-time solution for this problem. However, this does not mean that all nonpolynomial-time algorithms are equally “bad.” For example, here’s the simplest brute-force approach: For each permutation of the vertices, see if it forms a Hamiltonian path from $v_1$ to $v_n$. This takes time roughly proportional to $n!$, which is about $3\times10^{17}$ when $n = 20$.

Show that the Hamiltonian Path Problem can in fact be solved in time $O(2^n \cdot p(n))$, where $p(n)$ is a polynomial function of $n$. This is a much better algorithm for moderate values of $n$; for example, $2^n$ is only about a million when $n = 20$.

In addition, show that the Hamiltonian Path problem can be solved in time $O(2^n \cdot p(n))$ and in polynomial space.

\solution

\subsolution{High-level description}

Given $G=(V, E)$ with $V = \{v_1,v_2, \dots,v_n\}$, we want to decide if there is a Hamiltonian path from $v_1$ to $v_n$. We regard $H(G, v_1, v_n)$ as the solution of the question. Clearly, $H(G, v_1, v_n)$ is true if and only if there is a vertex $v_k$ with $H(G-\{v_1\}, v_k, v_n)$ is true, and there exist an edge $(v_1, v_k)$ so that we can get $H(G, v_1, v_n)=1$ by simply add the path $(v_1, v_k)$ to path $(v_k, v_n)$.

Following the idea, we can construct the answer starting from the largest sets and gradually working down to smaller ones. It's kind of like dynamic programming. 

The recursive formulation should be:
$$H(G, v_i, v_j) = \max_{k} H(G-\{v_i\}, v_k, v_j) \text{ and } (v_i, v_k) \in E$$

% \subsolution{Pseudo Code}

\subsolution{Correctness}

Base Case: The base case is the graph $G$ only contain 2 vertices $v_1$ and $v_2$, at least one edge exists between $v_1$ and $v_2$. Hence, $H(G, v_1, v_2)=1$.

The recursive formulation is correct because of that any Hamiltonian path from $v_i$ to $v_j$ can be divided into one Hamiltonian path $v_k$ to $v_j$ plus one edge $(v_i, v_k)$. Hence the recursive formulation is true. By considering all the potential vertices, and following the dynamic programming, we ensure the $H$ is correctly computed. 

Correctness proved. 

\subsolution{Time complexity}

We need $O(2^n)$ to decide the subset graph and $v_1$ and $v_n$. At each time step, we need $O(n)$ time to decide the value of $i$. And determining if $ H(G-\{v_i\}, v_k, v_j)=1$ takes $O(n)$. Thus, the total time complexity should be $O(2^n \cdot n^2)=O(2^n \cdot p(n))$.

Since we follows the idea of dynamic programming, we store each time step's calculated results $ H(G-\{v_i\}, v_k, v_j)$. Hence, after chosen $v_1$ and $v_n$, the space complexity should be $O(n)$. The total space complexity is $O(2^n \cdot n)$.

