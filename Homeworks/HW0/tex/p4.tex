\newpage
\problem{4: Pond sizes} % {10+10+10+20=50}

\problemdes

You have an integer matrix representing a plot of land, where the value at a location represents the height above sea level. A value of zero indicates water. A pond is a region of water connected vertically, horizontally, or diagonally. The size of a pond is the total number of connected water cells. Write a method to compute the sizes of all ponds in the matrix.

\solution

\subsolution{High-level description}

Although the problem description says it's an integer matrix, we can regard it as a connected graph. The connectivity depends on vertically, horizontally and diagonally adjacent. 

Our high-level idea is: We start with a position $(i, j)$, if it has a value of zero, then it indicates water. We first set $size=1$, then we would like to find if the adjacent region is filled with water too, which means we should check the 8 adjacent positions $[(i+1, j), (i-1, j), (i, j+1), (i, j-1), (i-1, j-1), (i-1, j+1), (i+1, j+1), (i+1, j-1)]$. Notice, we should check if the position is at the edge so that index will not have access violation. 

If the value of adjacent positions is also zero, we should increase the sizes of pond: $size+1$. Also, to avoid double counting, after we visited each position, we should use a mark so that we will not visit it again in the future. For this problem, we can simple set $M[i, j] = 1$. 

\subsolution{Pseudo Code}

\begin{algorithm}[]
  \caption{Pond Size}
  \KwIn{$M[1 \dots m, 1 \dots n]$}
  \KwOut{A list containing the sizes of all ponds}
  initial $res$ as an empty list \;
  \For{$i=1; i \le m; i++$}
  {
    \For{$j= 1; j \le n; j++$}
    {
        \If{$M[i,j]$ is $0$}
        {
            $size = 1$ \;
            $checkAdjacentPositions(M, i, j, size)$ \;
            add $size$ into $res$ \;
        }
    }
  }
  return $res$\;
\end{algorithm}

\begin{algorithm}[]
  \caption{$checkAdjacentPositions$}
  \KwIn{$M[1 \dots m, 1 \dots n], i, j, size$}
  \KwOut{None but updates size}
  $//$ check access violation\;
  \If{$i<1$ or $j<1$ or $i>m$ or $j>n$ or $M[i,j]$ is not $0$}
  {
    return \;
  }
  
  $//$ $M[i,j]$ is $0$\;
  $size = size+1$ \;
  $M[i,j]=0$\;
  
  $//$ check adjacent positions\;
  $//$ vertically\;
  $checkAdjacentPositions(M, i+1, j, size)$ \;
  $checkAdjacentPositions(M, i-1, j, size)$ \;
  $//$ horizontally\;
  $checkAdjacentPositions(M, i, j+1, size)$ \;
  $checkAdjacentPositions(M, i, j-1, size)$ \;
  $//$ diagonally\;
  $checkAdjacentPositions(M, i-1, j-1, size)$ \;
  $checkAdjacentPositions(M, i-1, j+1, size)$ \;
  $checkAdjacentPositions(M, i+1, j+1, size)$ \;
  $checkAdjacentPositions(M, i+1, j-1, size)$ \;

\end{algorithm}

\subsolution{Correctness}

Actually, for each position $(i, j)$, we use DFS to search for all children nodes of it. As depth first search going, we count $size$ and mark the children as visited. This ensures both termination and the completeness.

\subsolution{Time complexity}

We can easily draw the conclusion that in algorithm \textit{Pond Size}, the outer and inner loop creates $O(mn)$ time complexity, here m is the number of rows and n is the number of columns of $M$. For operations inside the inner loop, we should notice a fact that there is at most once visit for each position (since we mark $M[i, j]=1$ for those ponds and will visit it again in the next time).

Overall speaking, the total time complexity for this algorithm is $O(mn)$.

% \subsolution{Data Structure}
