\newpage
\problem{2: Business plan} % {10+10+10+20=50}

\problemdes

Consider the following problem. You are designing the business plan for a start-up company. You have identified $n$ possible projects for your company, and for, $1 \leq i \leq n$, let $c_i > 0$ be the minimum capital required to start the project $i$ and $p_i > 0$ be the profit after the project is completed. You also know your initial capital $C_0 > 0$. You want to perform at most $k$, $1 \leq i \leq n$, projects before the IPO and want to maximize your total capital at the IPO. Your company cannot perform the same project twice.

In other words, you want to pick a list of up to k distinct projects, $i_{1}, \ldots, i_{k^{\prime}}$ with $k^{\prime} \leq k$. Your accumulated capital after completing the project $i_j$ will be $C_{j}=C_{0}+\sum_{h=1}^{h=j} p_{i_{h}}$. The sequence must satisfy the constraint that you have sufficient capital to start the project $i_{j+1}$ after completing the first $j$ projects, i.e.,$C_j \geq c_{i_{j+1}}$ for each $j=0, \dots, k^{\prime}-1$.You want to maximize the final amount of capital, $C_{k^{\prime}}$.

\solution

\subsolution{High-level description}

We use the greedy algorithm to solve this problem.

According to the problem description, we have the following observations:

1. We can only perform at most $k$ projects before the IPO.

2. We want to maximize our total capital. Each project has a $p_i$ as the profit after the project is completed. Our accumulated capital after completing the project $i_j$ should be $C_j = C_0 + \sum_{h=1}^{h=j} p_{i_h}$.

3. We should have sufficient capital to start the project $i_{j+1}$ after completing the first $j$ projects.

Intuitively, if we are to write a greedy algorithm for a problem like this, we might first consider choosing the projects based on its profit, i.e., we choose the project that has the highest profit each time. However, we should still consider another thing: we should have sufficient capital to start this profitable project. We now formally describe our algorithm:

Sort projects based on its minimum capital required to start the project. When choosing the $j$-th project, we should choose the most profitable project among all the projects $i_{j+1}$ where $c_{i_{j+1}} \leq C_j$. After chosen $k$ projects, terminate and return the final capital as well as what we've chosen.

% Return the final capital and the sequence of project selected!

%\subsolution{Pseudo Code}


\subsolution{Correctness}

We use the exchange argument to prove the correctness of this algorithm.

Imagine an optimal solution OPT and our greedy solution is G. For ease of comparison, we sort projects in both solutions by increasing profits. For the purpose of comparison, we will consider projects to be equal if they have the same profit and start capital (so duplicate projects are equal to each other). Now for project $o_i$ and $g_i$, we account for the following two cases:

1. $o_i = g_i$: The solutions agree, so we can move on.

2. $o_i \neq g_i$: We argue that this is not possible. Suppose there exists an optimal solution OPT in which $o_i \neq g_i$. Then, because the greedy algorithm prioritizes projects with higher profit when the start capital is sufficient. And after the most profitable project has been done, the current capital $C_i$ should also be the highest, which will leave more choosing space for the next project. Hence, it is impossible for OPT (or any other solution) to have a higher capital than G. Therefore, this case can only lead to OPT to be worse than G. However, this would contradict our knowledge that OPT is optimal. By proof by contradiction, there is no optimal solution in which $o_i \neq g_i$.

Therefore, we have proven that any optimal solution is necessarily equal to the greedy solution. 

\subsolution{Time complexity}

The initial sort is $O(n\log n)$. Each iteration afterward can be done by $O(\log n)$ if we maintain a max-heap data structure to help us choose the most profitable project under the condition $c_{i_{j+1}} \leq C_j$. And we do that iteration for $k$ times, i.e., this takes $O(k)$ time complexity. 

Overall, our algorithm can be completed in $O(n\log n+ k\log n)=O(n\log n)$.







