\newpage
\problem{1: Maximum Length Chain of Subwords} % {10+10+10+20=50}

\problemdes

Input: We are given a set of $n$ distinct strings of length at most $k$ over a finite alphabet $\sigma$.

Output: A sequence of strings that form a chain under the (consecutive) subword relation; i.e., if the output is $w_{1}, w_{2}, \dots, w_{t}$ then we can write $w_{i+1}=u w_{i} v$ for some strings $u,v$.

Find a chain of maximum length.

\solution

\subsolution{High-level description}

We use dynamic programming to solve this problem. 

\textit{Subproblem}: We refer $T(n)$ as the maximum length of chain containing $w_n$. 

\textit{Base Case}: For a set of only one string, the solution is simply the string itself.

\textit{Iteration}: We can have the state transition function as follows:

$$T(n) = \max(T(i) + 1), \text{ where }w_{n}=u w_{i} v, 1\leq i <n$$

Since our algorithm should output a sequence of string chain. We need to store some additional values:

1. $maxLength$: the length of the maximum chain

2. $maxIndex$: the tail string's index of the maximum length chain

3. $P(n)$: the index $i$ that maximize $T(n)$

We update $maxLength$ and $maxIndex$ each time after we computed $T(n)$. With $maxLength$, $maxIndex$ and $P(n)$, we can backtrack our maximum length chain of subwords following the order

$$[w_{maxindex}, w_{P(maxindex)}, w_{P(P(maxindex))}, \dots]$$

To find out whether $w_{n}=u w_{i} v$, we need to define a new algorithm called $isSubsequence$. We can choose a pattern matching algorithm to do the string comparison. Here, we choose KMP algorithm and the time complexity is proportional to the length of the string, which is O(k).

%\subsolution{Pseudo Code}


\subsolution{Correctness}

\textit{Base Case}: A set of only one string trivially has only one option for its maximum length chain of subwords.

\textit{Induction Hypothesis}: Suppose all subproblems are correctly computed up to an arbitrary length of $k$.

\textit{Induction Step}: We would like to prove that if the induction hypothesis holds, then our algorithm's "iteration" component correctly computes the subproblem of the length of $k+1$. The following chain of logic should be sufficient:

1. Per the hypothesis, all the lengths smaller than $k$ are correct.

2. For string $w_{k+1}$, we have $T(k+1) = \max(T(i) + 1), \text{ where }w_{k+1}=u w_{i} v, 1 \leq i < k+1$. We exhaustively check relevant subproblems.

3. We choose the $T(i)$ that yields a max $T(k+1)$. Therefore, we choose the right option from among our previous calculated values and record the index $i$ as $P(k+1)=i$.

4. Thus, given that our hypothesis holds, our update correctly computes the length of $k+1$'s solutions.

Because our base case, hypothesis, and step are correct, our algorithm is proven correct with a proof by induction.



\subsolution{Time complexity}

The $T(n)$ loop will traverse $n$ strings, which takes $O(n)$ time. At each $T(n)$, we will iterate through $i, 1
\leq i < n$ elements, takes $O(n)$ time. When at each $w_i$, we do a KMP algorithm to find out whether $w_i$ is a subword of $w_n$, which will take O(k) time, and finding $\max(T(i) + 1)$ will only take $O(1)$.

Overall, our algorithm takes $O(kn^2)$.



