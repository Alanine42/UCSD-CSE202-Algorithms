\newpage
\problem{1: Maximum weight subtree} % {10+10+10+20=50}

\problemdes

The maximum weight subtree is as follows. You are given a tree $T$ together with (not necessarily positive) weights $w(i)$ for each node $i \in T$. A subtree of $T$ is any connected subgraph of $T$, (so a subtree is not necessarily the entire subtree rooted at a node). You wish to find a subtree of $T$ that maximizes $\sum_{i \in S} w(i)$. Design an efficient algorithm for solving this problem. Note that there is a linear (in the number of nodes of the tree) time algorithm for this problem. You can assume that for each node in the tree $T$, you are given a list of its children as well as the parent pointer (except for the root node).

\solution

\subsolution{High-level description}

Assume for each node of this tree, it at most has $m$ children nodes. Our solution is to do a tree traversal. At each node, we iterate all the children nodes and find their subtree value. Then the value of subtree rooted at the current node is equal to the sum of current node value and the subtree values of all $m$ children nodes. After getting the current subtree weight sum, we compare it with the stored maximum value and take the largest.

\subsolution{Pseudo Code}

\begin{algorithm}[]
  \caption{Maximum weight subtree}
  \KwIn{root node $root$}
  \KwOut{$maxWeightSum$}
  \If{$root$ is not valid}
  {
  	return $0$\;
  }
  initialize $maxWeightSum$ as the min value of the integer, say, $-\inf$\;
  $findMaxWeightSum(root, maxWeightSum)$\;
  return $maxWeightSum$\;
\end{algorithm}

\begin{algorithm}[]
  \caption{Find max weight sum}
  \KwIn{root node $root$, $maxWeightSum$}
  \KwOut{$maxWeightSum$}
  \If{$root$ is not valid}
  {
  	return $0$\;
  }
  weightSum = root.value\;
  \For{child in root.children}
  {
  	$weightSum = weightSum + findMaxWeightSum(child, maxWeightSum)$
  }
  $maxWeightSum = \max(maxWeightSum, weightSum)$
  return $weightSum$\;
\end{algorithm}


\subsolution{Correctness}

We update $maxWeightSum$ each time we use $findMaxWeightSum(root, maxWeightSum)$ function to find a maximum weight sum. Every time we calculate the weight sum, we compare it with the maximum weight sum we've seen so far.

The weight sum at each node is computed as root weight plus all the children's weights. There are two cases, 

1. We choose the root node. Then the maximum weight sum should be children's maximum weight sum plus node weight. We'll iterate through each child and add up the final weight sum, then compare the weigh sum with the maximum sum we've seen so far.

2. We don't choose the root node. Then the maximum weight sum is one of the children's maximum weight sum. We'll update the $maxWeightSum$ when we iterate the $findMaxWeightSum$ function to compute children's value.

Also, our algorithm terminates. It returns 0 for leaf's children node value if the node is a leaf node. Thus, it's correct.

\subsolution{Time complexity}

Since the tree traversal only visits each node once, hence the time complexity is $O(n)$, and $n$ is the number of nodes.

We can also interpret the complexity as the following recursion tree: 

$T(n) = m*T(n/m)+ c$.

And the second step is $T(n/m) = m*T(n/(m^2)) + c$, thus $T(n) = m^2*T(n/(m^2)) + 2c + c$ ...

And the last step is $T(n) = n*T(1) + c(1+m^1+...m^h)$, with h is the hight of the recursion tree $h = \log_m(n)$.

So the time complexity is $O(n*c + c*(m^(h+1)-1)/(m-1)) = O(n)$


